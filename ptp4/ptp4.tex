\documentclass{article}

\usepackage[utf8]{inputenc}
\usepackage[T1]{fontenc}
\usepackage{lmodern}
\usepackage{ngerman}

\begin{document}

\begin{center}
\large
PTP4 Zusammenfassung: Theoretische Quantenmechanik
\end{center}



Physik am Ende des 19. Jahrhunderts: 
Bekannte Wechselwirkungen ueberzeugend beschrieben: 
Graviation in klassischer Mechanik durch Newton, Lagrange, Hamilton
Elektromagnetismus durch Maxwell'sche Gleichungen

Ungeklaerte Fragen: 
\begin{itemize}
\item Widerspruch Galilei-Invarianz in kl. Mechanik (Geschwindigkeiten addiert) und Maxwell Elektrodynamik (Lichtgeschwindigkeit Obergrenze) aufgeloest durch Lorentz Invarianz in Einsteins spezieller Relativitaetstheorie
\item Stabilitaet der Atome (im Rutherford Modell) nicht erklaerbar
\item diskrete Spektrallinien nicht erklaerbar
\item Schwarzkoerperstrahlung nicht beschreibbar (UV-Katastrophe)
\end{itemize}

\emph{Hohlraumstrahlung}: 
Stehende Wellen im Hohlraum: Moden \\
Es sind $\frac{L}{\lambda}$ Wellen auf Strecke L moeglich


Anzahl abschaetzen:


Kugel ($V_{Kugel}=\frac{4}{3}*\pi*r^3$)


Zwei Polarisationsrichtungen: E und B Feld bringt Faktor zwei


Radius ist $\frac{L}{\lambda}$

$N(\lambda)  = 2*\frac{4}{3}*\pi*(\frac{L}{\lambda})^3$

\begin{itemize}
\item\emph{Dispersionsrelation:} $k = \frac{\omega}{c}$
\item\emph{Kreisfrequenz:} $\omega = 2*\pi*\nu$
\item\emph{Wellenlaenge:} $\lambda = \frac{2*\pi}{k}$
\end{itemize}


\end{document}