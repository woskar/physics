\documentclass{article}


\usepackage[applemac]{inputenc}
\usepackage[T1]{fontenc}
\usepackage{lmodern}
\usepackage{ngerman}
\usepackage{braket}
\usepackage{amssymb}

\begin{document}

\title{PTP4 Zusammenfassung \\ Theoretische Quantenmechanik \\ Professor Matthias Bartelmann}

\date{Sommersemester 2017\\Heidelberg}
\maketitle


Ende des 19. Jahrhunderts beschrieb Physik ueberzeugend die bekannten Wechselwirkungen:
\begin{itemize}
\item Graviation in klassischer Mechanik durch Newton, Lagrange, Hamilton
\item Elektromagnetismus durch Maxwell'sche Gleichungen
\item Thermodynamik
\end{itemize}

Ungeklaerte Fragen: 
\begin{itemize}
\item Widerspruch Galilei-Invarianz in kl. Mechanik (Geschwindigkeiten addiert) und Maxwell Elektrodynamik (Lichtgeschwindigkeit Obergrenze) aufgeloest durch Lorentz Invarianz in Einsteins spezieller Relativitaetstheorie
\item Stabilitaet der Atome (im Rutherford Modell) nicht erklaerbar
\item diskrete Spektrallinien nicht erklaerbar
\item Schwarzkoerperstrahlung nicht beschreibbar (UV-Katastrophe)
\end{itemize}

\emph{Hohlraumstrahlung}: 
Stehende Wellen im Hohlraum: Moden \\
Es sind $\frac{L}{\lambda}$ Wellen auf Strecke L moeglich


Anzahl abschaetzen: \\
Kugel ($V_{Kugel}=\frac{4}{3}*\pi*r^3$)\\
Zwei Polarisationsrichtungen: E und B Feld bringt Faktor zwei\\
Radius ist $\frac{L}{\lambda}$\\
$N(\lambda)  = 2*\frac{4}{3}*\pi*(\frac{L}{\lambda})^3$\\

\begin{itemize}
\item\emph{Relativistische Energie-Impuls-Beziehung:} $E = \sqrt{p^2c^2 + m^2c^4} $
\item\emph{Dispersionsrelation:} $k = \frac{\omega}{c}$
\item\emph{Kreisfrequenz:} $\omega = 2*\pi*\nu$
\item\emph{Wellenlaenge:} $\lambda = \frac{2*\pi}{k}$
\end{itemize}

Kommutator: $[\hat{A}, \hat{B}] := \hat{A}\hat{B} - \hat{B}\hat{A}$ \\

Einsoperator: $\hat{I}=\sum_n{\ket{a_n}\bra{a_n}} + \int{\ket{a}\bra{a}}da$ \\

Dichteoperator: $\hat{\rho}=\sum_n{p_n \ket{n}\bra{n}}$ \\

Zeitentwicklungsoperator: $\hat{U}(t, t_0)\ket{\psi(t_0)} = \ket{\psi(t)}$ \\

Zeitentwicklungsoperator: $\hat{U}(t) = exp(-\frac{i}{\hbar}\hat{H}t) $ \\

Heisenberg-Gleichung: $i \hbar \frac{d}{dt} \hat{A}_H = \left[ \hat{A}_H, \hat{H}_H \right] + i \hbar \left( \partial_t \hat{A} \right)_H $\\

Zeitabhaengiger Operator: $\hat{A}_H(t) := \hat{U}^{-1}(t, t_0) \hat{A} \hat{U}(t, t_0)$ \\

Translationsoperator: $\hat{T}_{\vec{a}} = exp \left( - \frac{i}{\hbar} \vec{a} \cdot \hat{\vec{p}} \right)$ \\

Dyson Reihe:  $\hat{U}(t, t_0) = T exp\left(-\frac{i}{\hbar} \int_{t_0}^t{\hat{H}(t')}dt' \right)$ \\

Wechselwirkungsbild:  $i \hbar \frac{d}{dt} \ket{\psi(t)}_I = \hat{V}_I \ket{\psi(t)}_I$ \\

Stoeroperator: $\hat{V}_I(t) := \hat{U}_0^{-1} \hat{V} \hat{U}_0$ \\

ToDo: \\

6 Axiome \\ 

Rabi-Oszillationen \\

Stoermatrix \\

$Ortsoperator\ \hat{x}=\left\{\begin{array}{ll} 
x & (Ortsdarstellung) \\
i \hbar \nabla_p & (Impulsdarstellung)
\end{array}\right.$ \\

$Impulsoperator\ \hat{p}=\left\{\begin{array}{ll}  
- i \hbar \nabla_x & (Ortsdarstellung) \\
p & (Impulsdarstellung)
\end{array}\right.$ \\

Kommutator Ort \& Impuls: $[ \hat{x}_i, \hat{p}_j] = - i \hbar \left[x_i, \partial_j\right] = i \hbar \delta_{ij}$ \\

Zeitunabhaengige Schroedinger-Gleichung: $\hat{H}\phi(x) = E \phi(x)$ \\

Randbedingungen \\

Energieeigenwerte des HOC: $E_n = \hbar \omega \left( n + \frac{1}{2} \right) $�\\

Absteigeoperator: $\hat{a} := \frac{1}{\sqrt{2}} (u + \partial_u)$ \\

Aufsteigeoperator: $\hat{a}^{\dagger} := \frac{1}{\sqrt{2}} (u - \partial_u)$ \\

Anzahloperator $\hat{N}$ \\

Exponentialfunktion hoch Wirkung \\

Pfadintegral \\

Paritaetsoperator: $\hat{P}\psi(\vec{x}) = \psi(-\vec{x}) $ \\

Translationsoperator \\

Kugelflaechenfunktionen \\

Zeitumkehroperator $\hat{\mathcal{T}}\psi(\vec{x}, t) = \psi^*(\vec{x}, -t)$\\

Wasserstoffatom \\

Leiteroperatoren \\

Radiale Eigenfunktion des Wasserstoffatoms \\

Hamilton im elektromagnetischen Feld \\

Ahanorov-Bohm-Effekt \\

Zeeman-Effekt \\

Spinzustaende \\

Spinoren \\

Spinpraezession \\








\end{document}