\documentclass{article}

\usepackage[utf8]{inputenc}
\usepackage[T1]{fontenc}
\usepackage{lmodern}
\usepackage{ngerman}

\begin{document}

\begin{center}
\large
PTP4 Zusammenfassung: Theoretische Quantenmechanik
\end{center}


Physik am Ende des 19. Jahrhunderts: 
Bekannte Wechselwirkungen ueberzeugend beschrieben: 
Graviation in klassischer Mechanik durch Newton, Lagrange, Hamilton
Elektromagnetismus durch Maxwell'sche Gleichungen

Ungeklaerte Fragen: 
Widerspruch Galilei-Invarianz in kl. Mechanik (Geschwindigkeiten addiert) und Maxwell Elektrodynamik (Lichtgeschwindigkeit Obergrenze) aufgeloest durch Lorentz Invarianz in Einsteins spezieller Relativitaetstheorie

Stabilitaet der Atome (im Rutherford Modell) nicht erklaerbar
diskrete Spektrallinien nicht erklaerbar
Schwarzkoerperstrahlung nicht beschreibbar (UV-Katastrophe)

Hohlraumstralung: 
Stehende Wellen im Hohlraum: Moden
L/lambda Wellen auf Strecke L moeglich

Anzahl abschaetzen:
Kugel (4/3 pi r\^3)
zwei Polarisationsrichtungen: E und B Feld bringt Faktor zwei
Radius ist L/lambda 

N(lambda)  = 2 * 4/3 * pi * (L/lambda) \^3

Dispersionsrelation: k = omega/c
Kreisfrequenz: omega = 2*pi*nu
Wellenlaenge: lambda = 2*pi/k



\end{document}