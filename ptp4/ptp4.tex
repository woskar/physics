\documentclass{article}


\usepackage[applemac]{inputenc}
\usepackage[T1]{fontenc}
\usepackage{lmodern}
\usepackage{ngerman}
\usepackage{braket}
\usepackage{amssymb}

\begin{document}

\title{PTP4 Zusammenfassung \\ Theoretische Quantenmechanik \\ Professor Matthias Bartelmann}

\date{Sommersemester 2017\\Heidelberg}
\maketitle


Ende des 19. Jahrhunderts beschrieb Physik ueberzeugend die bekannten Wechselwirkungen:
\begin{itemize}
\item Graviation in klassischer Mechanik durch Newton, Lagrange, Hamilton
\item Elektromagnetismus durch Maxwell'sche Gleichungen
\item Thermodynamik
\end{itemize}

Ungeklaerte Fragen: 
\begin{itemize}
\item Widerspruch Galilei-Invarianz in kl. Mechanik (Geschwindigkeiten addiert) und Maxwell Elektrodynamik (Lichtgeschwindigkeit Obergrenze) aufgeloest durch Lorentz Invarianz in Einsteins spezieller Relativit\"atstheorie
\item Stabilitaet der Atome (im Rutherford Modell) nicht erkl\"arbar
\item diskrete Spektrallinien nicht erkl\"arbar
\item Schwarzk\"orperstrahlung nicht beschreibbar (UV-Katastrophe)
\end{itemize}

\emph{Hohlraumstrahlung}: 
Stehende Wellen im Hohlraum: Moden \\
Es sind $\frac{L}{\lambda}$ Wellen auf Strecke L m\"oglich


Anzahl absch\"atzen: \\
Kugel ($V_{Kugel}=\frac{4}{3}*\pi*r^3$)\\
Zwei Polarisationsrichtungen: E und B Feld bringt Faktor zwei\\
Radius ist $\frac{L}{\lambda}$\\
$N(\lambda)  = 2*\frac{4}{3}*\pi*(\frac{L}{\lambda})^3$\\

\begin{itemize}
\item\emph{Relativistische Energie-Impuls-Beziehung:} $E = \sqrt{p^2c^2 + m^2c^4} $
\item\emph{Dispersionsrelation:} $k = \frac{\omega}{c}$
\item\emph{Kreisfrequenz:} $\omega = 2*\pi*\nu$
\item\emph{Wellenl\"ange:} $\lambda = \frac{2*\pi}{k}$
\end{itemize}


Freie Schr\"odingergleichung: $i \hbar \frac{\partial}{\partial t} \psi(t, \vec{x}) = -\frac{\hbar^2}{2m}\vec{\nabla}^2\psi(t, \vec{x}) $ \\

Energieoperator: $\hat{E} = i \hbar \frac{\partial}{\partial t} $ \\

Kommutator: $[\hat{A}, \hat{B}] := \hat{A}\hat{B} - \hat{B}\hat{A}$ \\

Einsoperator: $\hat{I}=\sum_n{\ket{a_n}\bra{a_n}} + \int{\ket{a}\bra{a}}da$ \\

Dichteoperator: $\hat{\rho}=\sum_n{p_n \ket{n}\bra{n}}$ \\

Zeitentwicklungsoperator: $\hat{U}(t, t_0)\ket{\psi(t_0)} = \ket{\psi(t)}$ \\

Zeitentwicklungsoperator: $\hat{U}(t) = exp(-\frac{i}{\hbar}\hat{H}t) $ \\

Heisenberg-Gleichung: $i \hbar \frac{d}{dt} \hat{A}_H = \left[ \hat{A}_H, \hat{H}_H \right] + i \hbar \left( \partial_t \hat{A} \right)_H $\\

Zeitabh\"angiger Operator: $\hat{A}_H(t) := \hat{U}^{-1}(t, t_0) \hat{A} \hat{U}(t, t_0)$ \\

Translationsoperator: $\hat{T}_{\vec{a}} = exp \left( - \frac{i}{\hbar} \vec{a} \cdot \hat{\vec{p}} \right)$ \\

Dyson Reihe:  $\hat{U}(t, t_0) = T exp\left(-\frac{i}{\hbar} \int_{t_0}^t{\hat{H}(t')}dt' \right)$ \\

Wechselwirkungsbild:  $i \hbar \frac{d}{dt} \ket{\psi(t)}_I = \hat{V}_I \ket{\psi(t)}_I$ \\

St\"oroperator: $\hat{V}_I(t) := \hat{U}_0^{-1} \hat{V} \hat{U}_0$ \\

6 Axiome \\ 
\begin{itemize}
\item Zust\"ande werden durch repr\"asentiert durch Strahlen im Hilbertraum
\item Observablen entsprechen linearen selbstadjungierten Operatoren
\item Eigenwerte sind m\"ogliche Messwerte
\item Entwicklungskoeffizientenquadrate (Zust\"ande auf Basisvektoren projiziert) sind Wahrscheinlichkeiten f\"ur Messung
\item Zeitentwicklung durch Schr\"odingergleichung
\item Durch den Messprozess wird der Zustandsvektor identisch mit einem der Eigenbasisvektoren des Operators welcher der Messung entspricht
\end{itemize}

Rabi-Oszillationen \\

St\"ormatrix \\

$Ortsoperator\ \hat{x}=\left\{\begin{array}{ll} 
x & (Ortsdarstellung) \\
i \hbar \nabla_p & (Impulsdarstellung)
\end{array}\right.$ \\

$Impulsoperator\ \hat{p}=\left\{\begin{array}{ll}  
- i \hbar \nabla_x & (Ortsdarstellung) \\
p & (Impulsdarstellung)
\end{array}\right.$ \\

Kommutator Ort \& Impuls: $[ \hat{x}_i, \hat{p}_j] = - i \hbar \left[x_i, \partial_j\right] = i \hbar \delta_{ij}$ \\

Zeitunabh\"angige Schr\"odinger-Gleichung: $\hat{H}\phi(x) = E \phi(x)$ \\

Randbedingungen: stetiges Potential: zweimal diffbare WF \\

HOC Absteigeoperator: $\hat{a} := \frac{1}{\sqrt{2}} (u + \partial_u)$ \\

HOC Aufsteigeoperator: $\hat{a}^{\dagger} := \frac{1}{\sqrt{2}} (u - \partial_u)$ \\

HOC Besetzungszahloperator $\hat{N} := \hat{a}^{\dagger} \hat{a} $ \\

HOC Hamilton: $ \hat{H} = \hbar \omega \left( \hat{N} + \frac{1}{2} \right) $ \\

HOC Energieeigenwerte: $ E_n = \hbar \omega \left( n + \frac{1}{2} \right)$\\

HOC Energieeigenzust\"ande: $\ket{n} = \frac{1}{\sqrt{n!}}(\hat{a}^{\dagger})^n \ket{0}$ \\

Exponentialfunktion hoch Wirkung \\

Pfadintegral \\

Parit\"atsoperator: $\hat{P}\psi(\vec{x}) = \psi(-\vec{x}) $ \\

Translationsoperator \\

Kugelfl\"achenfunktionen \\

Zeitumkehroperator $\hat{\mathcal{T}}\psi(\vec{x}, t) = \psi^*(\vec{x}, -t)$\\

Wasserstoffatom \\

Leiteroperatoren \\

Radiale Eigenfunktion des Wasserstoffatoms \\

Hamilton im elektromagnetischen Feld \\

Ahanorov-Bohm-Effekt \\

Zeeman-Effekt $\Delta E = - \frac{m \hbar \omega_B}{2} = B \mu_B m$ \\

Spinzust\"ande \\

Spinoren \\

Spinpr\"azession $\ket{s(t)} = exp(i \omega_L t \sigma_3) \ket{s(0)} $ \\

Larmorfrequenz $\omega_L := \frac{g}{2 \hbar} \mu_B B$ \\

Paschen-Back-Effekt \\

St\"orung \\

Linearer Stark-Effekt \\

Legendre Transformation $ \mathcal{L}(q, \dot{q}, t) = p\dot{q} - H(q, p, t) $\\

Wirkung $S = \oint \mathcal{L} dt  = \oint pdq - Et = S_0 - Et $ \\

Quantisierungsregel: $S_0 = n \cdot h$ \\

Fermis Goldene Regel $\Gamma = \frac{2\pi}{\hbar} \left| \bra{\psi_m^{0}} \hat{H}^{(1)} \ket{\psi_n^{(0)}} \right|^2 \rho(E_n) $\\

Virialsatz $2 \left< T \right> = \left< \hat{x} \cdot \vec{\nabla}V(\hat{x}) \right> $ \\

Virialsatz f\"ur homogenes Potential vom Grad k ist $2 \left< T \right> = k \left< V \right> $ \\

Gauss-Integral $ \int_{-\infty}^{\infty}x^2e^{-ax^2} dx = \frac{1}{2a} \sqrt{\frac{\pi}{a}}$ \\

Pauliverbot \\

Variationsverfahren \\

WKB N\"aherung \\

Formfaktor \\

Optisches Theorem \\


\end{document}